\documentclass[11pt,a4paper]{article}
\usepackage{amsmath}
\usepackage{amsfonts}
\usepackage{amssymb}
\usepackage{enumitem}

\usepackage{geometry}
\geometry{a4paper,left=2cm,right=2cm,top=1cm,bottom=1cm}

\begin{document}
	\begin{center}
		Data Science
		
		Ruichun Liu
		
		Problem Set 8
	\end{center}
\begin{enumerate}

% latex table generated in R 3.5.3 by xtable 1.8-4 package
% Fri Mar 20 17:26:55 2020
\begin{table}[ht]
\centering
\begin{tabular}{rrrrrrrr}
  \hline
 & beta & beta\_OLS & beta\_GD & beta\_OLS\_L & beta\_OLS\_NM & beta\_MLE & beta\_LM \\ 
  \hline
X1 & 1.50 & 1.50 & 1.50 & 1.50 & 1.50 & 1.13 & 1.50 \\ 
  X2 & -1.00 & -0.99 & -0.99 & -0.99 & -0.99 & -1.09 & -0.99 \\ 
  X3 & -0.25 & -0.25 & -0.25 & -0.25 & -0.25 & -0.30 & -0.25 \\ 
  X4 & 0.75 & 0.74 & 0.74 & 0.74 & 0.74 & 0.76 & 0.74 \\ 
  X5 & 3.50 & 3.50 & 3.50 & 3.50 & 3.50 & 3.73 & 3.50 \\ 
  X6 & -2.00 & -2.00 & -2.00 & -2.00 & -2.00 & -2.18 & -2.00 \\ 
  X7 & 0.50 & 0.50 & 0.50 & 0.50 & 0.50 & 0.47 & 0.50 \\ 
  X8 & 1.00 & 1.00 & 1.00 & 1.00 & 1.00 & 1.00 & 1.00 \\ 
  X9 & 1.25 & 1.26 & 1.26 & 1.26 & 1.26 & 1.31 & 1.26 \\ 
  X10 & 2.00 & 2.00 & 2.00 & 2.00 & 2.00 & 2.14 & 2.00 \\ 
   \hline
\end{tabular}
\end{table}


    \item \textbf{Question 5} How does your estimate compare with the true value of $\beta$ in (1)?

		  \begin{description}
	            From the column beta\_OLS, we can see that my estimates are very close to the true value.      
          \end{description}
          
    \item \textbf{Question 7} Do your answers differ?
		  \begin{description}
               From the column beta\_OLS\_L and beta\_OLS\_NM, we can see that the results of two different methods are the exactly same.         
          \end{description}
          
    \item \textbf{Question 9} In your .tex file, tell me
          about how similar your estimates of $\beta$ are to the “ground truth”  that you used to
          create the data in (1).

		  \begin{description}
               From the table 1 on next page, we can find that each of the variables is significant under $ p<0.01$.        
          \end{description}
          
          % Table created by stargazer v.5.2.2 by Marek Hlavac, Harvard University. E-mail: hlavac at fas.harvard.edu
% Date and time: Fri, Mar 20, 2020 - 5:49:58 PM
\begin{table}[!htbp] \centering 
  \caption{} 
  \label{} 
\begin{tabular}{@{\extracolsep{5pt}}lc} 
\\[-1.8ex]\hline 
\hline \\[-1.8ex] 
 & \multicolumn{1}{c}{\textit{Dependent variable:}} \\ 
\cline{2-2} 
\\[-1.8ex] & Y \\ 
\hline \\[-1.8ex] 
 X1 & 1.501$^{***}$ \\ 
  & (0.002) \\ 
  & \\ 
 X2 & $-$0.991$^{***}$ \\ 
  & (0.003) \\ 
  & \\ 
 X3 & $-$0.247$^{***}$ \\ 
  & (0.003) \\ 
  & \\ 
 X4 & 0.744$^{***}$ \\ 
  & (0.003) \\ 
  & \\ 
 X5 & 3.504$^{***}$ \\ 
  & (0.003) \\ 
  & \\ 
 X6 & $-$1.999$^{***}$ \\ 
  & (0.003) \\ 
  & \\ 
 X7 & 0.502$^{***}$ \\ 
  & (0.003) \\ 
  & \\ 
 X8 & 0.997$^{***}$ \\ 
  & (0.003) \\ 
  & \\ 
 X9 & 1.256$^{***}$ \\ 
  & (0.003) \\ 
  & \\ 
 X10 & 1.999$^{***}$ \\ 
  & (0.003) \\ 
  & \\ 
\hline \\[-1.8ex] 
Observations & 100,000 \\ 
R$^{2}$ & 0.971 \\ 
Adjusted R$^{2}$ & 0.971 \\ 
Residual Std. Error & 0.500 (df = 99990) \\ 
F Statistic & 338,240.000$^{***}$ (df = 10; 99990) \\ 
\hline 
\hline \\[-1.8ex] 
\textit{Note:}  & \multicolumn{1}{r}{$^{*}$p$<$0.1; $^{**}$p$<$0.05; $^{***}$p$<$0.01} \\ 
\end{tabular} 
\end{table} 
          
\end{enumerate}
\end{document}
