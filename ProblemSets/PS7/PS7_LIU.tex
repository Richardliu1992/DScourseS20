\documentclass[11pt,a4paper]{article}
\usepackage{amsmath}
\usepackage{amsfonts}
\usepackage{amssymb}
\usepackage{enumitem}
\usepackage{graphicx}

\usepackage{geometry}
\geometry{a4paper,left=2cm,right=2cm,top=1cm,bottom=1cm}

\begin{document}
	\begin{center}
		Data Science
		
		Ruichun Liu
		
		Problem Set 7
	\end{center}
\begin{enumerate}



    \item \textbf{Question 6}
		  \begin{description}
	            % Table created by stargazer v.5.2.2 by Marek Hlavac, Harvard University. E-mail: hlavac at fas.harvard.edu
% Date and time: Tue, Mar 10, 2020 - 11:51:59 PM
\begin{table}[!htbp] \centering 
  \caption{} 
  \label{} 
\begin{tabular}{@{\extracolsep{5pt}}lccccccc} 
\\[-1.8ex]\hline 
\hline \\[-1.8ex] 
Statistic & \multicolumn{1}{c}{N} & \multicolumn{1}{c}{Mean} & \multicolumn{1}{c}{St. Dev.} & \multicolumn{1}{c}{Min} & \multicolumn{1}{c}{Pctl(25)} & \multicolumn{1}{c}{Pctl(75)} & \multicolumn{1}{c}{Max} \\ 
\hline \\[-1.8ex] 
logwage & 1,686 & 1.622 & 0.388 & 0.005 & 1.358 & 1.936 & 2.261 \\ 
hgc & 2,244 & 13.099 & 2.521 & 0.000 & 12.000 & 15.000 & 18.000 \\ 
tenure & 2,231 & 5.978 & 5.510 & 0.000 & 1.583 & 9.333 & 25.917 \\ 
age & 2,246 & 39.153 & 3.060 & 34 & 36 & 42 & 46 \\ 
\hline \\[-1.8ex] 
\end{tabular} 
\end{table}

% Table created by stargazer v.5.2.2 by Marek Hlavac, Harvard University. E-mail: hlavac at fas.harvard.edu
% Date and time: Tue, Mar 10, 2020 - 11:55:17 PM
\begin{table}[!htbp] \centering 
  \caption{} 
  \label{} 
\begin{tabular}{@{\extracolsep{5pt}}lccccccc} 
\\[-1.8ex]\hline 
\hline \\[-1.8ex] 
Statistic & \multicolumn{1}{c}{N} & \multicolumn{1}{c}{Mean} & \multicolumn{1}{c}{St. Dev.} & \multicolumn{1}{c}{Min} & \multicolumn{1}{c}{Pctl(25)} & \multicolumn{1}{c}{Pctl(75)} & \multicolumn{1}{c}{Max} \\ 
\hline \\[-1.8ex] 
logwage & 1,669 & 1.625 & 0.386 & 0.005 & 1.362 & 1.936 & 2.261 \\ 
hgc & 2,229 & 13.101 & 2.524 & 0 & 12 & 15 & 18 \\ 
tenure & 2,229 & 5.971 & 5.507 & 0.000 & 1.583 & 9.333 & 25.917 \\ 
age & 2,229 & 39.152 & 3.062 & 34 & 36 & 42 & 46 \\ 
\hline \\[-1.8ex] 
\end{tabular} 
\end{table}

From the table two, we can see that there are 1669 variables for logwage and 2229 \\for each others. Therefore, the missing rate is about 25 \%. I think it is MNAR.
          \end{description}
          
    \item \textbf{Question 7}
		  \begin{description}
		  Table 3 is on the next page. Since the true value is 0.093, we can see the results from complete cases and predicted value cases are quit close to the true value. The result from mean imputation is small. \\
		  Results in column 1 and 3 are the same. I think the reason might be that case 3 uses the predicated value from case 1, which would not change the value of coefficient. However, mean imputation will lead to a lower beta value.  Missing Completely At Random and Missing At Random are good ways to predict the model.\\
		  Beta 1 in column 2 means that logwage would increase by 4.9\%, if there were 100\% increase in hgc. Beta 1 in column 3 means that logwage would increase by 6.2\%, if there were 100\% increase in hgc.

% Table created by stargazer v.5.2.2 by Marek Hlavac, Harvard University. E-mail: hlavac at fas.harvard.edu
% Date and time: Wed, Mar 11, 2020 - 12:01:10 AM
\begin{table}[!htbp] \centering 
  \caption{} 
  \label{} 
\begin{tabular}{@{\extracolsep{5pt}}lccc} 
\\[-1.8ex]\hline 
\hline \\[-1.8ex] 
 & \multicolumn{3}{c}{\textit{Dependent variable:}} \\ 
\cline{2-4} 
\\[-1.8ex] & \multicolumn{3}{c}{logwage} \\ 
\\[-1.8ex] & (1) & (2) & (3)\\ 
\hline \\[-1.8ex] 
 hgc & 0.062$^{***}$ & 0.049$^{***}$ & 0.062$^{***}$ \\ 
  & (0.005) & (0.004) & (0.004) \\ 
  & & & \\ 
 collegenot college grad & 0.146$^{***}$ & 0.160$^{***}$ & 0.146$^{***}$ \\ 
  & (0.035) & (0.026) & (0.025) \\ 
  & & & \\ 
 tenure & 0.023$^{***}$ & 0.015$^{***}$ & 0.023$^{***}$ \\ 
  & (0.002) & (0.001) & (0.001) \\ 
  & & & \\ 
 age & $-$0.001 & $-$0.001 & $-$0.001 \\ 
  & (0.003) & (0.002) & (0.002) \\ 
  & & & \\ 
 marriedsingle & $-$0.024 & $-$0.029$^{**}$ & $-$0.024$^{*}$ \\ 
  & (0.018) & (0.014) & (0.013) \\ 
  & & & \\ 
 Constant & 0.639$^{***}$ & 0.833$^{***}$ & 0.639$^{***}$ \\ 
  & (0.146) & (0.115) & (0.111) \\ 
  & & & \\ 
\hline \\[-1.8ex] 
Observations & 1,669 & 2,229 & 2,229 \\ 
R$^{2}$ & 0.195 & 0.132 & 0.268 \\ 
Adjusted R$^{2}$ & 0.192 & 0.130 & 0.266 \\ 
Residual Std. Error & 0.346 (df = 1663) & 0.311 (df = 2223) & 0.300 (df = 2223) \\ 
F Statistic & 80.508$^{***}$ (df = 5; 1663) & 67.496$^{***}$ (df = 5; 2223) & 162.884$^{***}$ (df = 5; 2223) \\ 
\hline 
\hline \\[-1.8ex] 
\textit{Note:}  & \multicolumn{3}{r}{$^{*}$p$<$0.1; $^{**}$p$<$0.05; $^{***}$p$<$0.01} \\ 
\end{tabular} 
\end{table} 
              
          \end{description}    
          
    \item \textbf{Question 8}
		  \begin{description}

Now I am working on a project about education in China. My topic is how education funding would affect the acceptance rate of high school in different regions and provinces in China. I have collected education funding data, the acceptance rate of high school, population, GDP, income, etc. during 1996-2017. I am going to use the panel data to run a FE Model (at province or city unit level). I would like to see how education funding affects the acceptance rate and if there are significant differences in different regions and provinces.
 
          \end{description} 

          
\end{enumerate}
\end{document}
